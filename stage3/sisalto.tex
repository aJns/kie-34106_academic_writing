I chose to read the paper ``Health care applications: a solution based on the
internet of things''~\cite{Bui2011}, which is one of the papers I've chosen to
use in my writing assignment on IoT applications in healthcare. In reading the
paper, I tried to follow the advice of the linked text ``How to Read a Paper''
by Keshav \& Cheriton. They recommend reading the paper in three passes, each
of which increases the detail that is paid to the text.

In the first pass I read the headings, abstract, introduction and references.
This gave a pretty good overall view on the paper. It also allowed me to answer
the ``five Cs'':

\begin{enumerate}
  \item Category: My chosen paper is sort of a cross between an analysis of the
    current state of the field, and a proposal for its future implementation.

  \item Context: The context of the paper is a meeting point between multiple
    different fields. There's the electronics and sensors which are used to
    collect data, the communication networks that stream and collect that
    data and there's also some mention of data mining, processing and handling.

  \item Correctness: The assumptions that the paper makes appear valid, but
    that's partly due to the fact that the paper doesn't make a lot of
    assumptions or conclusions, it just mostly follows the current state of
    affairs, with a little futuristic predictions thrown in.

  \item Contributions: The contributions are giving a clear view of the state
    of the field, which is actually very useful for me, or others who are
    interested but new to the field. It also tries to propose ways of how these
    systems utilizing IoT in healthcare can be formed, but the effects of that
    can only be seen when they are formed.

  \item Clarity: I felt it was written clearly enough. There are some long,
    meandering sentences with a lot of difficult words, but mostly it was easy
    enough to understand.

\end{enumerate}

In the second pass I read all the sections that I didn't read before. I also
studied the figures carefully. The text used a hypothetical person in the
near-future, and followed his day and how it was made better by IoT healthcare
applications. After that they detailed the existing technology and protocols,
and how they could be best used to construct a working IoT healthcare system.

I didn't need a third pass, as the text wasn't that long, and it was quite
simple to understand. I also knew enough about the technologies presented that
I didn't feel the need to read up on them.

I think the most useful part of this exercise was trying out the ``three passes
method''. Especially as ``newbie'' in reading articles its very helpful to have
a structured way to read them. The method also seems great that you can quickly
and easily ``pre-process'' papers to see if they're worth reading. Definitely
helpful for the future, and something I'll intend to use in writing my theses.
