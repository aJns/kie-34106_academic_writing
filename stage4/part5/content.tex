The advances in technology have shaped, and continue to shape our lives
immensely in all areas of life. The internet has revolutionized communication
in our everyday lives, and it's slowly beginning to change the way we approach
healthcare as well.

A fairly new area that's being brought on by the internet
is the Internet of Things, usually abbreviated as IoT or IOT. IoT means
increased connection of all our devices. This enables them to monitor and
collect massive amounts of data about our lives and our environment, which in
turn gives us whole new ways to solve the problems that face us, like the
problems in healthcare.

The internet has already improved healthcare in some ways; Patient treatment
history and plans can be easily shared between doctors and clinics over online
databases, and statistic insights can be found from a collection of
(anonymized) treatment information.

The true potential of these advancements can be had when they're combined. IoT
devices, for example wearable bands or implanted sensors, can gather data from
the patient at all times. When they're connected to the internet, they can send
this data to databases, where the patient's doctor can easily see and analyze
it. This can reveal trends that are wholly invisible today, and all without the
patient needing come to the clinic. The devices can also alert the medical
authorities when they detect that the patient is in critical condition.

An important but often overlooked part of this vision of the future is the
requirement of open standards. It's critical that all parts of the system use
the same interfaces, and that is only possible if manufacturers can freely use
them.


Source:~\cite{Bui2011}
