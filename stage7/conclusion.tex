\section{Conclusions}
\label{sec:conclusion}

This literature review looks at the way that IoT technologies can improve and
in some ways revolutionize our health care. IoT applications in health care can
provide a higher quality of care with reduced costs and manpower. In some
cases, IoT applications can remove the need for costly and time consuming, and
sometimes impossible visits to the doctor. They can also enable patients with
chronic conditions, the elderly and people with disabilities to live at home
independently, instead of at the hospital or a nursing home. 

There's still need for more research and development in IoT and its
applications in the healthcare domain. Open and widely adopted standards are
needed to enable devices and software systems from different manufacturers to
work together. States and institutions have to digitize and organize their data
to enable IoT applications to reach their full potential. Sharing of patient
records has to be uniform, simple and confidential. IoT and software security
has to be taken into account at every step, and safety critical systems have to
carefully designed. Fortunately there's a lot of interest and investment put
into IoT, so research will continue to improve the technology in the
foreseeable future.
