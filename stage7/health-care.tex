\section{IoT applications in health care}
\label{sec:iot-health}

Because IoT is in itself a term that encompasses many things, and because it
has many applications in health care, it's beneficial to approach the subject
by dividing its applications into a few different categories.

IoT research and applications in the health care domain can be divided into
different categories depending on the point of view. For example the in the
article~\cite{Islam2015} the authors divide the research of IoT in health care
in to three different ``levels'' of research:

\begin{description}
  \item[Topology] is concerned with the actual devices and their connections
    and tasks. Topology research might for example try to figure out how to
    transfer the data from a patient's wearable to the cloud for processing.

  \item[Architecture] research is more focused on the organization of devices
    and the software and protocol they use. In architecture research an
    important goal might be defining what protocol the devices use to transfer
    the data.

  \item[Platform] research tries to create a good environment for IoT
    operation and development. It does this by developing standards and
    reusable software libraries.

\end{description}

All of these research fields are concerned with IoT applications in health
care, but they differ on which part of the ecosystem they research.

IoT applications can also be categorized by what kind of service they aim to
provide. For example, the challenges and requirements in treating a certain
chronic afliction can be very different than trying to look out for early
symptoms in preventive care. Age and other factors that vary between patients
might also pose different needs. For example, older patients are not as
technologically inclined, and thus need extremely clear and simple to use
devices. Again, the article~\cite{Islam2015} defines many categories by
service. We'll take a look at a few of the different service types:


% TODO: check the below descriptions using real sources
\begin{description}
  \item[Ambient assisted living or AAL] is meant to enable elderly or otherwise
    disabled individuals to live independently for longer and/or more easily
    than currently possible. AAL uses most of the key technologies of IoT. As
    the patients are elderly or disabled, the devices should be very easy to
    use, or better yet, practically invisible to the user. An example AAL
    architecture might include a wireless sensor network that monitors the
    patient. In addition to that, the patient might have some sensors implanted
    on their body, or they might use a wearable device to monitor them. The
    data collected from the sensors and wearables would be analyzed by an AI
    using machine learning methods. Using that data the AI could then
    independently administer the required medicine, or it could notify the
    doctor of changes in the patient's status. Using voice and facial
    recognition, the AI could for example open doors for the patient, operate
    other devices or fetch items, all of which could be hard for the patient to
    do on their own. Using IoT the patients relatives could also easily check
    on them and keep in touch.~\cite{Istepanian2011}

  \item[Adverse drug reaction or ADR] is an injury caused by medication. It can
    be caused by improper dosage, a combination of drugs or a prior condition,
    for example an unnoticed allergy. IoT can be used to prevent an adverse
    drug reaction. One such system is detailed in the article~\cite{Jara2010}.
    In that system, the patient has a terminal which can identify the drug to
    be taken by means of NFC or barcode. The terminal then checks an electronic
    health record which has the patient's treament info, and determines if the
    drug is safe to take. If the drug might trigger an allergy, or if it could
    be dangerous combined with another drug that the patient is taking, it
    warns the patient.

  \item[Indirect emergency healthcare or IEH] 

  \item[Community healthcare] 

\end{description}
