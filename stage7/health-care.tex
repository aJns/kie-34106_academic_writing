\section{IoT applications and challenges in health care}
\label{sec:iot-health}

IoT has many applications in healthcare. It can be used in monitoring
chronic illnesses, preventive health care and alerting emergency personnel in
case of accidents or acute diseases. The article~\cite{Bui2011} describes a
hypothetical scenario where a patient is treated with the aid of an IoT system.
The system can monitor his general health, and alert him and his doctor if it
detects changes that might warrant a checkup. It also helps the patient monitor
his diet, and when an accident happens it alerts an ambulance to the patient's
location.

IoT research and applications in the health care domain can be divided into
different categories depending on the point of view. For example the in the
article~\cite{Islam2015} the authors divide the research of IoT in health care
in to three different ``levels'' of research:

\begin{description}
  \item[Topology] is concerned with the actual devices and their connections
    and tasks. Topology research might for example try to figure out how to
    transfer the data from a patient's wearable to the cloud for processing.

  \item[Architecture] research is more focused on the organization of devices
    and the software and protocol they use. In architecture research an
    important goal might be defining what protocol the devices use to transfer
    the data.

  \item[Platform] research tries to create a good environment for IoT
    operation and development. It does this by developing standards and
    reusable software libraries.

\end{description}

All of these research fields are concerned with IoT applications in health
care, but they differ on which part of the ecosystem they research.

IoT applications can also be categorized by what kind of service they aim to
provide. For example, the challenges and requirements in treating a certain
chronic affliction can be very different from trying to look out for early
symptoms in preventive care. Age and other factors that vary between patients
might also pose different needs. For example, older patients are not as
technologically inclined, and thus need extremely clear and simple to use
devices. Again, the article~\cite{Islam2015} defines many categories by
service. We'll take a look at a few of the different service types:

\begin{description}
  \item[Ambient assisted living or AAL] is meant to enable elderly or otherwise
    disabled individuals to live independently for longer and/or more easily
    than currently possible. AAL uses most of the key technologies of IoT. As
    the patients are elderly or disabled, the devices should be very easy to
    use, or better yet, practically invisible to the user. An example AAL
    architecture might include a wireless sensor network that monitors the
    patient. In addition to that, the patient might have some sensors implanted
    on their body, or they might use a wearable device to monitor them. The
    data collected from the sensors and wearables would be analyzed by an AI
    using machine learning methods. Using that data the AI could then
    independently administer the required medicine, or it could notify the
    doctor of changes in the patient's status. With voice and facial
    recognition technology, the AI could for example open doors for the
    patient, operate other devices or fetch items, all of which could be hard
    for the patient to do on their own. IoT technologies can also enable the
    patient's relatives to easily check on them and keep in
    touch.~\cite{Istepanian2011}

  \item[Adverse drug reaction or ADR] is an injury caused by medication. It can
    be caused by improper dosage, a combination of drugs or a prior condition,
    for example an unnoticed allergy. IoT can be used to prevent an adverse
    drug reaction. One such system is detailed in the article~\cite{Jara2010}.
    In that system, the patient has a terminal which can identify the drug to
    be taken by means of NFC or bar code. The terminal then checks an electronic
    health record which has the patient's treatment info, and determines if the
    drug is safe to take. If the drug might trigger an allergy, or if it could
    be dangerous combined with another drug that the patient is taking, it
    warns the patient.

  \item[Community healthcare] is a concept where a network is created to cover
    a community or area. Often the area or community in question is a rural
    one, where access to health care might be very limited. A community
    healthcare system would enable the patients to be served by a ``virtual
    hospital''. Their health status could be monitored remotely, and they could
    get access to medical advice without traveling long distances. The
    article~\cite{Rohokale2011} proposes such a system, which is especially
    designed to be operated in developing countries. The system is energy
    efficient and robust, because all the nodes can communicate and transfer
    data through each other.

\end{description}

IoT applications in health care also face many challenges. Many are problems
that are faced in all IoT application domains, but some are especially
important to take into account in the health care domain. As many of the
proposed applications are safety critical, they have to be very reliable and be
able to handle outages. Because of this, and also because health care systems
have a lot of confidential data, the security of IoT systems has to carefully
considered. While there's really no good ``one size fits all'' approach to IoT
security, basic things to consider are confidentiality, authentication and
availability. Availability in the security context means that the system should
continue working at an adequate level even if it's under a denial-of-service
attack. Authentication means ensuring the identity of other nodes in the
network, so that confidential data isn't sent to an attacker posing as a node.
This also requires good encryption of the data transfer. Finally,
confidentiality means that the system only allows access to the data by parties that
are allowed to see it, like the doctor and the patient.~\cite{Islam2015} All
these challenges are aggravated by general IoT requirements; IoT devices
usually have to be mobile, energy efficient and small, which constrains the
solutions available.






















