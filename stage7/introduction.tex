\section{Introduction}

In recent years the concept of IoT (internet of things) has been gaining a lot
of interest, especially in the field of electronics and computer science.
Simply stated, the internet of things means the increased linking of embedded
systems with each other and the internet. Embedded systems mean the electronics
within devices, buildings and vehicles, as opposed to the general computing
devices like computers and smart phones that we use every day.

The reason IoT has been gaining as much interest as it has is because it
enables automation and collection of data on a whole new level. This can mean
tracking inventories accurately in real time, or optimizing heating and air
conditioning depending on how many people are at home and in which rooms. It's
not hard to imagine that this kind of monitoring would be immensely useful in
health care. Doctors wouldn't have to rely on a 30 minute visit to make a
diagnosis, instead they could check the data collected during the week to see
if there are signs of illness.

There are many reasons to adopt IoT technologies in health care. In many
western countries to population is aging rapidly, without the population growth
needed to care for that aging population. IoT technologies can take reduce the
workload on nurses and doctors, while also enabling more independent living for
the elderly. In younger and healthier populations, IoT can be used to improve
the quality of preventive healthcare through increased monitoring. By
preventing serious health problems from forming, the costs of health care are
reduced and patient life quality improved.~\cite{Bui2011}

To better understand the healthcare applications of IoT, I first aim to define IoT and
explore the technologies that enable it Section~\ref{sec:iot}. After that we'll
look at how IoT can be applied tackle challenges in the health care domain in
Section~\ref{sec:iot-health}.  Finally I present my conclusions in
Section~\ref{sec:conclusion}.
