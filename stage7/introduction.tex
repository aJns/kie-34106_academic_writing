\section{Introduction}

In recent years the concept of IoT (internet of things) has been gaining a lot
of interest, especially in the field of electronics and software development.
Simply stated, the internet of things means the increased linking of embedded
systems with each other and the internet. Embedded systems mean the electronics
within devices, buildings and vehicles, as opposed to the general computing
devices like computers and smart phones that we use every day.

The reason IoT has been gaining as much interest as it has is because it
enables automation and collection of data on a whole new level. This can mean
tracking inventories accurately in real time, or optimizing heating and air
conditioning depending on how many people are at home and in which rooms. It's
not hard to imagine that this kind of monitoring would be immensely useful in
health care. Doctors wouldn't have to rely on a 30 minute visit to make a
diagnosis, instead they could check the data collected during the week to see
if there are signs of illness.

To better understand the applications of IoT, I first aim to define IoT and
explore the technologies that enable it section~\ref{sec:iot}. Then I lay out
the challenges in health care that IoT technologies can help solve in
section~\ref{sec:health}. After that I look at how IoT can tackle those
challenges in section~\ref{sec:iot-health}.  Finally I present my conclusions
in section~\ref{sec:conclusion}.
