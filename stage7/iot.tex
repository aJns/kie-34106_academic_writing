\section{The internet of things}
\label{sec:iot}

The term ``internet of things'' or IoT has become quite a buzzword in recent
years. One consequence of achieving buzzword status is that the term can become
a bit unclear, which is somewhat true with IoT;\@ Although the high level
concept and basics are pretty clear, a lot of researchers have their own
opinion on how wide the term is, and what technologies are encompassed by it
and which are separate from it. In this section I'll try to explain the concept
in a way that I think most people working in electronics or computer science
can agree on.

In my opinion the predecessor of the modern concept of IoT was and is the idea
of ``wireless sensor networks'' or WSNs. The idea of WSNs is to deploy large
numbers of simple and cheap networked smart sensors that effectively monitor an
area. As the military applications of that kind of technology are self-evident,
US military research agency DARPA sponsored programs researching sensor
networks for use on the battlefield. Back then the sensors were usually
stationary. Today the applications of WSNs have thankfully broadened to
civilian fields, like rescue and environmental monitoring. Modern WSN are also
increasingly mobile, the sensors being carried by robots or
drones.~\cite{Chong2003}

So WSNs are one part of the internet of things. Put simply, they're the eyes
and ears of the concept, they collect the data. But collecting the data is not
enough, somebody or something also has to make sense of it. Machine learning
methods can take the massive amounts of data that WSNs generate, and pick out
patterns and trends from it. They can predict future conditions form present
ones, or identify and track people from images using computer
vision.~\cite{Gubbi2013} To be clear though, machine learning is separate
concept from IoT, and it has a multitude of other fields where it can be
applied. However, machine learning is vital to the successful application of
IoT.

The third part that makes IoT possible is the improvement of communications
technologies, especially the improvement of wireless communications. Most IoT
applications would be impractical if they were constrained by physical wires.
The internet of things also benefits immensely from open standards and
protocols like IP and HTTP that have been widely adopted because of the
internet. This enables products from different manufacturers to work seamlessly
together.

Modern communications technologies combined with increasing computing power in
small chips has also enabled distributed computing.  Distributed computing
means that instead of sending all the data to a central server for processing,
a WSN for example can have multiple (less powerful) processing nodes in the
network. This lessens the strain that data transfer has on the network, and
also makes the processing more robust because there isn't a single point of
failure.~\cite{Yu2006}

So to sum up, IoT consists of data collection, connectivity and processing.
Another part that we haven't mentioned is actuating, or acting upon the
environment instead of just monitoring it. It might be a considered a big part
of IoT in the future, but at the moment a lot of research on IoT isn't focused
on it, and it seems that most researchers don't consider it a part of the
concept.
